\textbf{Theorem.} Suppose $p$ and $q$ are primes with $q < p$ and $q$ does not divide
$p - 1$. Then any group $G$ of order $pq$ is cyclic.

\textbf{Proof.}
Assume the statement is false. By Lagrange's Theorem we may assume that every
nonidentity element of $G$ has order $p$ or $q$. Let $a \ne e$ belong to $G$.
If $|C(a)| = pq$, then for any $b \notin \langle a \rangle$ we have $|b| \ne |a|$,
for otherwise $|\langle a, b\rangle| = p^{2}$ or $q^{2}$. But $|b| \ne |a|$ and
$b \in C(a)$ implies $|ab| = pq$. So, we may assume that for all nonidentity
elements $a$ in $G$, $|C(a)| = p$ or $q$.

We now count the elements of order $p$ and $q$. Since $|a| = p$ implies
$|\mathrm{cl}(a)| = |G|/|C(a)| = q$, the number of elements of order $p$ is a
multiple of $q$. Moreover, because $|a| = p$ implies $|a^{i}| = p$ for
$i = 2, \dots, p-1$, the number of elements of order $p$ is also a multiple of
$p - 1$. Since $\gcd(q, p - 1) = 1$, the number of elements of order $p$ is a
multiple of $q(p - 1)$. Analogously, the number of elements of order $q$ is a
multiple of $p(q - 1)$. Since neither $q(p - 1)$ nor $p(q - 1)$ divides $pq$,
not all the nonidentity elements of $G$ can have the same order, thus there
must be at least $q(p - 1) + p(q - 1) > pq$ elements in $G$. This contradiction
finishes the proof.
